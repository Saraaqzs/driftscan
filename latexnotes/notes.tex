
\documentclass{revtex4}

\usepackage{amsmath}
\usepackage{mathtools}
\newcommand{\la}{\ensuremath{\left\langle}}
\newcommand{\ra}{\ensuremath{\right\rangle}}

\usepackage{tikz}
\usetikzlibrary{arrows,decorations.pathmorphing,backgrounds,positioning,fit,petri}
\begin{document}

\title{KL-notes}

\renewcommand{\vec}[1]{\mathbf{#1}}
\newcommand{\mat}[1]{\mathbf{#1}}

\newcommand{\hconj}{\ensuremath{\dagger}}

\newcommand{\vx}{\vec{x}}
\newcommand{\vs}{\vec{s}}
\newcommand{\vn}{\vec{n}}
\newcommand{\va}{\vec{a}}
\newcommand{\vv}{\vec{v}}

\newcommand{\mS}{\mat{S}}
\newcommand{\mN}{\mat{N}}
\newcommand{\mR}{\mat{R}}
\newcommand{\mC}{\mat{C}}
\newcommand{\mB}{\mat{B}}

\newcommand{\tcm}{\ensuremath{21\,\mathrm{cm}}}

\newcommand{\vnhat}{\hat{\vec{n}}}
\newcommand{\vu}{\vec{u}}

\newcommand{\brsc}[1]{{\ensuremath{\scriptscriptstyle \left(#1\right)}}}

\maketitle


\section{Formalism}

\subsection{Unpolarised Case}

This is based on a combination of the notation of Albert and Ue-Li,
I've just ripped off some of their notation. I'm keeping this
unpolarised for the time being, however this is a really bad
approximation for cylinders.

A particular feed $F_i$ measures a combination of the electric field
$E(\vnhat)$ coming from various directions on the sky.
\begin{equation}
F_i = \int d^2\vnhat \, A_i(\vnhat) E(\vnhat) e^{i \vnhat\cdot\vu_i}\; ,
\end{equation}
where the function $A_i(\vnhat)$ gives the amplitude of the beam in
that direction, and the exponential factor keeps track of the phase
relative to an arbitrary reference point, and $\vu_i$ is the
displacement to that point, divided by the wavelength.. We assume that the
radiation from the sky is \emph{incoherent}, that is
\begin{equation}
\la E(\vnhat) E^* (\vnhat') \ra = \delta(\vnhat - \vnhat') I(\vnhat) \; .
\end{equation}

A visibility $V_{ij}$ is the correlation between two feeds $F_i$
and $F_j$ at a particular instant
\begin{align}
V_{ij} & = \la F_i F_j^* \ra \\
&= \int d^2\vnhat \, A_i(\vnhat) A_j(\vnhat) e^{i \vnhat \cdot
  \vu_{ij}} I(\vnhat)
\end{align}
where $\vu_{ij} = \vu_i - \vu_j$. For brevity of notation, we will
introduce an index $\alpha$ which represents individual visibilities,
or pairs $ij$ of feeds. We will write the above equation can be written in terms of a
transfer function $B_\alpha$, and will introduce Earth rotation by
adding a dependence on the azimuthal angle $\phi$. We also introduce a
term for the noise common to both feeds 
\begin{equation}
V_\alpha(\phi) = \int \! d^2\vnhat\, B_\alpha(\vnhat; \phi) I(\vnhat)
+ n_\alpha(\phi)
\end{equation}
where
\begin{equation}
B_\alpha(\vnhat; \phi) = A_\alpha^2(\vnhat; \phi) e^{i \vnhat \cdot
  \vu_{\alpha}(\phi)} \; .
\end{equation}
As expected in the rotating system the $UV$-plane changes orientation
with respect to the celestial sphere. Given the periodicity of the
system, Fourier transforming the system is an obvious next step
\begin{align}
V^\alpha_m &= \int \frac{d\phi}{2\pi} V_\alpha(\phi) e^{-i m \phi} \\
& = \sum_{l m'}\int \frac{d\phi}{2\pi} B^\alpha_{l m'}(\phi) a^{T*}_{l
  m'}  e^{-i m \phi}+ n^\alpha_m
\end{align}
where we have to proceed to the second line we have inserted the
multipole expansions of both the sky, and the beam transfer
function. As the $\phi$ dependence simply rotates the functions about
the polar axis $B^\alpha_{lm}(\phi) =
B^{\alpha}_{lm}(\phi=0)$. Combined with the exponential factor in the
integral, this simply generates the Kroenecker delta $\delta_{mm'}$,
and we find
\begin{equation}
V^\alpha_m = \sum_{l} B^\alpha_{l m} a^{T*}_{l m}+ n^\alpha_m \; .
\end{equation}

\subsection{Polarised Case}

Extending to the polarised case seems not to be too difficult, it just
requires some careful generalisation of the above case. Again taking
many cues from Albert.

In this case a feed is generalised, with the index referring not just
to the position on the telescope but the polarisation too
\begin{equation}
F_i(\phi) = \int d^2\vnhat \, A^a_i(\vnhat; \phi) E_a(\vnhat) e^{i
  \vnhat\cdot\vu_i(\phi)}\; .
\end{equation}
In this the index summation over $a$ accounts for the relative angles
of the polarisation and feed. In most cases it should be possible to
seperate the function $A_i^a$ into an angular reception pattern, and a
vector representing the alignment of the dipole for the feed
\begin{equation}
A_i^a(\vnhat; \phi) = A_i(\vnhat; \phi) \hat{d}^a_i \; .
\end{equation}
The radiation from the sky is still \emph{incoherent}, but now we must
include its polarisation (neglecting circular polarisation)
\begin{equation}
\la E_a(\vnhat) E_b^* (\vnhat') \ra = \left[\mathcal{P}_{ab}^I
  I(\vnhat) + \mathcal{P}_{ab}^Q Q(\vnhat) + \mathcal{P}_{ab}^U U(\vnhat)\right]\delta(\vnhat - \vnhat') \; ,
\end{equation}
where the tensor $\mathcal{P}^I_{ab} = I_{ab}$, the identity, and pulls
out the total intensity. The $\mathcal{P}^Q_{ab}$, and
$\mathcal{P}^U_{ab}$ pull out the standard $Q$ and $U$ polarisation
states, and are equal to the third and first Pauli matrices
respectively. In summary
\begin{equation}
\mathcal{P}^I_{ab} = \begin{pmatrix} 1 & 0 \\ 0 & 1\end{pmatrix} \; ,
\qquad
\mathcal{P}^Q_{ab} = \begin{pmatrix} 1 & 0 \\ 0 & -1\end{pmatrix} \; ,
\qquad
\mathcal{P}^U_{ab} = \begin{pmatrix} 0 & 1 \\ 1 & 0\end{pmatrix} \; ,
\end{equation}
in an orthonormal basis. The standard basis to use in spherical
geometry are the polar and azimuthal directions, $\hat{\theta}$ and
$\hat{\phi}$, as these allow spin spherical harmonics to be used
straightforwardly to decompose the polarisation field.

A visibility has the same definition as before, but now we need to
take into account the polarisations of the feeds, and the sky
\begin{equation}
\label{eq:vispol}
V_{ij}(\phi) = \int d^2\vnhat \, \left[ B^T_{ij}(\vnhat; \phi) I(\vnhat) +
  B^Q_{ij}(\vnhat; \phi) Q(\vnhat + B^U_{ij}(\vnhat; \phi) U(\vnhat \right]+ n_{ij}(\phi)
\end{equation}
where the beam transfer functions $B^X_{ij}$ are
\begin{equation}
B^X_{ij}(\vnhat; \phi) = A_i^a(\vnhat; \phi) A_j^{b *}(\vnhat; \phi)
\mathcal{P}^X_{ab} \:e^{i \vnhat \cdot
  \vu_{ij}(\phi)} 
\end{equation}
To proceed we need to insert the spherical harmonic decompositions,
however as polarisation is not a scalar field, we must expand in
spin-2 harmonics $Y_{lm}^\brsc{\pm 2}(\vnhat)$
\begin{align}
I(\vnhat) & = \sum_{lm} a^T_{lm} Y_{lm}(\vnhat) \; ,\\
Q(\vnhat) + i U(\vnhat) & = \sum_{lm} a^\brsc{+2}_{lm}
Y^\brsc{+2}_{lm}(\vnhat) \; ,\\
Q(\vnhat) - i U(\vnhat) & = \sum_{lm} a^\brsc{-2}_{lm}
Y^\brsc{-2}_{lm}(\vnhat) \; .
\end{align}
In actual fact we will insert the complex conjugate of the above
expressions ($I$, $Q$ and $U$ are real). Presuming that the polarised beam
transfer matrices also transform as spin fields (pretty sure, should
check), we decompose them in the same way, with 
\begin{align}
B^Q_{\alpha}(\vnhat) + i B^U_{\alpha}(\vnhat) & = \sum_{lm}
B^\brsc{+2}_{\alpha;lm} (\phi)
Y^\brsc{+2}_{lm}(\vnhat) \; ,\\
B^Q_{\alpha}(\vnhat) - i B^U_{\alpha}(\vnhat) & = \sum_{lm}
B^\brsc{-2}_{\alpha;lm} (\phi)
Y^\brsc{-2}_{lm}(\vnhat) \; .
\end{align}
where for brevity we have again switched to using $\alpha$.

Combined with the orthogonality of the (spin) spherical harmonics this
allows us to rewrite the visibility equation \eqref{eq:vispol} as
\begin{equation}
V_{\alpha}(\phi) = \sum_{lm} \left[B_{\alpha;lm}^T(\phi) a^{T*}_{lm} +
\frac{1}{2} B_{\alpha;lm}^\brsc{+2}(\phi) a^{\brsc{+2}*}_{lm} +
\frac{1}{2} B_{\alpha;lm}^\brsc{-2}(\phi) a^{\brsc{-2}*}_{lm}\right] + n_{\alpha}(\phi)
\; .
\end{equation}
Though this has thoroughly transformed the problem into harmonic
space, it will be more convenient if we further transform into the
conventional $E$ and $B$ mode decomposition. This can be done by
making the standard substitutions
\begin{align}
a^\brsc{+2}_{lm} & = -\left(a^E_{lm} + i a^B_{lm}\right) \; ,\\
a^\brsc{-2}_{lm} & = -\left(a^E_{lm} - i a^B_{lm}\right)
\end{align}
as well as the corresponding changes for the beam matrices
\begin{align}
B^\brsc{+2}_{\alpha;lm} & = -\left(B^E_{\alpha;lm} + i B^B_{\alpha;lm}\right) \; ,\\
B^\brsc{-2}_{\alpha;lm} & = -\left(B^E_{\alpha;lm} - i
  B^B_{\alpha;lm}\right) \; .
\end{align}
This leaves the visibility as
\begin{equation}
V_{\alpha}(\phi) = \sum_{lm} \left[B_{\alpha;lm}^T(\phi) a^{T*}_{lm} +
B_{\alpha;lm}^E(\phi) a^{E*}_{lm} + B_{\alpha;lm}^B(\phi)
a^{B*}_{lm}\right] + n_{\alpha}(\phi)
\; .
\end{equation}
In the above the harmonic coefficients are now all the transforms of
real scalar (or pseudo-scalar) fields. All that is left is to perform
the Fourier transform in $\phi$, which proceeds as in the unpolarised
case to give
\begin{equation}
V_{\alpha; m} = \sum_{l} \left[B_{\alpha;lm}^T a^{T*}_{lm} +
B_{\alpha;lm}^E a^{E*}_{lm} + B_{\alpha;lm}^B
a^{B*}_{lm}\right] + n_{\alpha; m}
\; .
\end{equation}

introduce an index $\alpha$ which represents individual visibilities,
or pairs $ij$ of feeds. We will write the above equation can be written in terms of a
transfer function $B_\alpha$, and will introduce Earth rotation by
adding a dependence on the azimuthal angle $\phi$. We also introduce a
term for the noise common to both feeds 
\begin{equation}
V_\alpha(\phi) = \int \! d^2\vnhat\, B_\alpha(\vnhat; \phi) I(\vnhat)
+ n_\alpha(\phi)
\end{equation}
where
\begin{equation}
B_\alpha(\vnhat; \phi) = A_\alpha^2(\vnhat; \phi) e^{i \vnhat \cdot
  \vu_{\alpha}(\phi)} \; .
\end{equation}
As expected in the rotating system the $UV$-plane changes orientation
with respect to the celestial sphere. Given the periodicity of the
system, Fourier transforming the system is an obvious next step
\begin{align}
V^\alpha_m &= \int \frac{d\phi}{2\pi} V_\alpha(\phi) e^{-i m \phi} \\
& = \sum_{l m'}\int \frac{d\phi}{2\pi} B^\alpha_{l m'}(\phi) a^T_{l
  m'}  e^{-i m \phi}+ n^\alpha_m
\end{align}
where we have to proceed to the second line we have inserted the
multipole expansions of both the sky, and the beam transfer
function. As the $\phi$ dependence simply rotates the functions about
the polar axis $B^\alpha_{lm}(\phi) =
B^{\alpha}_{lm}(\phi=0)$. Combined with the exponential factor in the
integral, this simply generates the Kroenecker delta $\delta_{mm'}$,
and we find
\begin{equation}
V^\alpha_m = \sum_{l} B^\alpha_{l m} a^T_{l m}+ n^\alpha_m \; .
\end{equation}

\section{Signal-Noise Eigenmodes}

% Define a measurement in some vector space
% \begin{equation}
% x^i = s^i + n^i
% \end{equation}
% where $s^i$ and $n^i$ are respectively the signal we are interested in and some
% form of noise. These components have covariance matrices
% \begin{equation}
% \la s^i s^j \ra = S^{ij}, \qquad \la n^i n^j \ra = N^{ij} \; .
% \end{equation}
% Let us start with the generalised eigenvalue equation
% \begin{equation}
% S^{ij} e_j^\alpha = \lambda N^{ij} e_j^\alpha
% \end{equation}
% As $N$ is symmetric positive definite, lets insert its diagonalisation
% $N^{ij} = f^i_a N^{ab} f^j_b$ into the above equation.



Define a measurement in some, possibly complex, vector space
\begin{equation}
\vx = \vs + \vn
\end{equation}
where $\vs$ and $\vn$ are respectively the signal we are interested in and some
generalised form of noise (for instance in the case of \tcm this may
include foregrounds). These components have covariance matrices
\begin{equation}
\la \vs \vs^\hconj\ \ra = \mS, \qquad \la \vn \vn^\hconj \ra = \mN \; .
\end{equation}
We are free to transform the measurement vector as we wish $\vx' = \mR\vx$, provided
we are careful to update any related quantities. In the case of
gaussian distributed measurements it is sufficient to transform the
covariance matrix $\mat{X}' = \la (\mR \vx) (\mR \vx)^\hconj \ra  =
\mR \mat{X}\mR^\hconj$. The Karhunen-Loeve (KL) transform takes
advantage of this to produce simultaneous eigenmodes of the signal and
noise covariances.

Start by making the eigendecomposition of the noise matrix
\begin{equation}
\mN = \mR_1^\hconj \mN' \mR_1
\end{equation}
where $\mR_1$ is the unitary matrix of eigenvectors (stacked row by
row), and $\mN'$ is the diagonal matrix of eigenvectors. Using this we
can transform the data vector $\vx' = \mR_1 \vx$, which produces a new
signal covariance
\begin{equation}
\mS' = \la \vs' \vs'^\hconj \ra = \la (\mR_1 \vs)  (\mR_1 \vs)^\hconj
\ra = \mR_1 \mS \mR_1^\hconj
\end{equation}
and reduces the noise matrix to $\mN'$. As the new noise matrix
consists solely of positive diagonal elements $(\mN')_{ii} =
\lambda^N_i$, a further transformation $\vx'' = \mR_2 \vx'$, where
$\mR_2 = \mN'^{-\frac{1}{2}}$, reduces the noise matrix to the identity
$\mN'' = \mat{I}$. The signal matrix is transformed to
\begin{equation}
\mS'' = \mR_2 \mR_1 \mS \mR_1^\hconj \mR_2^\hconj \; .
\end{equation}
Applying any unitary transformation to the data will leave the noise
covariance as the identity. We use this freedom to diagonalise the
signal covariance by eigendecomposition $\mS'' = \mR_3^\hconj \mC
\mR_3$, leaving the total transformation on the data as
\begin{equation}
\vx = \mR_3 \mR_2 \mR_1 \vx
\end{equation}

Let us start with the generalised eigenvalue equation
\begin{equation}
S^{ij} e_j^\alpha = \lambda N^{ij} e_j^\alpha
\end{equation}
As $N$ is symmetric positive definite, lets insert its diagonalisation
$N^{ij} = f^i_a N^{ab} f^j_b$ into the above equation.



\section{Ue-Li's Talk}

\newcommand{\vlrarrow}{\ensuremath{\xrightarrow{\qquad\quad}}}

Our desire when mapping \tcm{} intensity is to obtain quantities like
the angular diameter distance $d_A$ and the Hubble parameter $H(z)$ as
a function of redshift. This needs several stages, and importantly
means propagating the errors through all of them
\begin{equation}
T(\theta, \nu) \xrightarrow[\text{thermal noise }\vn]{B_\alpha(\theta,\nu)} \quad \begin{matrix} \\ \\V_\alpha^\nu \\ \\ \mN\end{matrix} \quad
\vlrarrow \quad a_{lm}^\nu  \quad \vlrarrow \quad \begin{matrix} \\ \\P(k, \mu; z) \\ \\ \mC_P \end{matrix}  \quad \vlrarrow \quad \begin{matrix} \\ \\  d_A,\: H(z)  \\ \\ \sigma_d, \; \sigma_H \end{matrix} \; .
\end{equation}
The observation takes the sky $T(\theta, \nu)$ and maps it onto a set
of visibilities $V_\alpha^\nu$ through the beam transfer function
$B_\alpha(\theta, \nu)$ with some added thermal noise $\vn$. The index
$\alpha$ refers to the set of all measurements, that is baselines,
polarisations etc. Taking into account the statistics of the noise
$\mN$ this must be reduced down to a map of the sky, from which we can
remove any foregrounds and this is then propagated into a power
spectrum $P(k, \mu; z)$ (with a covariance $\mC_P$) from which the
interesting quantites $d_A(z)$, $H(z)$ and their errors can be
deduced.

The sky itself can be decomposed into spherical harmonics
\begin{equation}
T(\theta, \nu) = \sum_{lm} Y_{lm}(\theta) a_{lm}^\nu \; .
\end{equation}
If we are using a transit telescope such as CHIME, the observations
are periodic in the rotation of the Earth, provided the telescope has
been `rigidised' giving stationnary noise $\vn$. This allows us to
decompose the visibilities $V_\alpha^{\nu}(\theta)$ (which are a
function of right-ascension $\phi$) onto a periodic basis
\begin{equation}
V_\alpha^m = \int V_\alpha(\phi) e^{i m \phi} d\phi \; .
\end{equation}
This set of visibility modes is the sum the sky, beamed onto the
visibility, plus the instrumental noise added into the
mode. Decomposing the beam into spherical harmonics, this is written as
\begin{equation}
V_\alpha^{m,\nu} = \sum_{l \geq m} B_{\alpha,l}^{\nu,m} a_{lm}^\nu + n_\alpha^m \; .
\end{equation}
We want to extract the sky $a_{lm}$ from this, and we can do this by
looking at the least squares solution
\begin{equation}
\tilde{\va} = {\underbracket{(\mB^\hconj \mN^{-1} \mB)}_{\text{singular}}}^{-1} \mB^\hconj \mN^{-1} \vv \; .
\end{equation}
As we expect the quantity in brackets to be singular this motivates
the use of the dirty map
\begin{equation}
\tilde{\va}^0 = \mB^\hconj \mN^{-1} \vv \; .
\end{equation}
What exactly is the noise matrix $\mN$? If we were simply interested
in mapping the sky it would simply be the covariance of the
instrumental noise, however as we are interested in the \tcm {} signal it
is actually the covariance of the data with this set to zero and so it
should include galactic foregrounds
\begin{equation}
N_{\alpha \alpha'} = \la V_\alpha V_{\alpha'} \ra \rvert_{s = 0}  = \delta_{\alpha \alpha'} \sigma_n^2 + F_{\alpha \alpha'}
\end{equation}
where $F_{\alpha\alpha'}$ is the foreground covariance transfered into
the observed space. In the above we have omitted the $m$ and $\nu$
indices for clarity.

For simplicity let's assume that the foregrounds are isotropic (depend
only on $l$), and are perfectly correlated in frequency, then they can
be written as
\begin{equation}
\la a^f_l a^f_{l'} \ra = C_l^f \delta_{l l'} \delta_{m m'} F(\nu) F(\nu') \; .
\end{equation}

The beam matrix transfers the foreground correlations on the sky into their noise contribution
\begin{equation}
N_{\alpha \alpha'} = \delta_{\alpha \alpha'} \sigma_n^2 + \underbracket{B_{\alpha l} C_l^f B_{\alpha' l}^\hconj}_{F_{\alpha \alpha'}} \; .
\end{equation}
To diagonalise the foreground contribution $\mat{F} = \mS
\mat{\Lambda} \mS^\hconj$ we need to rotate the space of visibilities
$\vec{w} = \mS^\hconj \vv$. As the instrumental noise component is the
identity matrix (multiplied by $\sigma_n^2$), it is unchanged by this
transformation. This leaves a new noise covariance in this basis
\begin{equation}
\mN = \mat{I} \sigma_n^2 + \begin{pmatrix} \lambda_\text{max} & & 0 \\ & \ddots & \\ 0 & & \lambda_\text{min}
\end{pmatrix} \; .
\end{equation}
From this covariance matrix we can see that if $\lambda_\text{min} \gg
\sigma_n^2$, all modes are dominated by foregrounds, and the signal
can never be seen with the instrumental precision. As the olny freedom
we have is in the beam $B_\alpha$, the experiment must be designed so
that this is not inevitable.

\begin{figure}
\begin{tikzpicture}

\draw (0,0) -- (4,0) -- (4,6) -- (0,6) -- cycle;
\draw (4.1,0) -- (8.1,0) -- (8.1,6) -- (4.1,6) -- cycle;

\draw[fill=black] (2,3) circle(0.1);
\draw[fill=black] (6.1,4) circle(0.1);

\draw[<->] (0,-0.5) -- (4,-0.5)
node [below,midway] {$W$};

\draw[dashed] (0,3) -- (10,3);
\draw[dashed] (0,4) -- (10,4);

\draw[<->] (9.5,3) -- (9.5,4)
node [right,midway] {$D$};

\draw[dashed] (2,0) -- (2,8);
\draw[dashed] (6.1,0) -- (6.1,8);

\draw[<->] (2,7) -- (6.1,7)
node [above,midway] {$W+\epsilon$};


\end{tikzpicture}
\caption{A pair of cylinders (infintely long) of width $W$ and
  separation $\epsilon$. We consider a pair of dipoles one on each
  cylinder, such that they are separated by $W+\epsilon$ In the E-W
  direction and by $D$ in the N-S direction.}
\label{fig:2cylinder}
\end{figure}

In Figure~\ref{fig:2cylinder} a dipole pair is considered on two
cylinders. These correspond to a position in the u-v place of $u = W /
\lambda$ and $v = D / \lambda$ (ignoring the small cylinder separation
$\lambda$ for now). In terms of spherical harmonics coefficients on
the sky, points in the u-v plane map to $m \sim 2\pi u$ and $l \sim
\sqrt{m^2 + (2\pi v)^2}$. However because of the finite cylinder width
(and separation) the visibilities are sensitive to a range in $m$
\begin{equation}
V_{m,\nu} \qquad \frac{2 \pi \epsilon}{\lambda} < m < \frac{4 \pi
  W}{\lambda}
\end{equation}
which corresponds to the closest, and furthest points of each
cylinder. This means that when we are considering quantities like the
beam transfer $B_{\alpha,l}^{m,\nu}$ we only need consider a finite
range in $m$ as we can't possibly be sensitive to higher values.
\end{document}